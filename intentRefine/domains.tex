
\documentclass[11pt, oneside]{article}   
\usepackage{geometry}   
\geometry{a4paper}                   		
\usepackage{graphicx}				
\usepackage{float}

\usepackage{amssymb, amsmath,mathtools}
\usepackage{wrapfig}

\usepackage{multirow}
\usepackage{hhline}

\begin{document}
\section{Eigth Domains with increasingly higher complexity} 
\label{sec:expres}
\begin{enumerate}
\item Complexity level \textbf{L1:} only one object with one fine-resolution part; two rooms
  with one cells in each room.

\item Complexity level \textbf{L2:} two objects with two fine-resolution parts
  each; three rooms with two cells each room.

\item Complexity level \textbf{L3:} three objects with three fine-resolution parts
  each; four rooms with four cells in each room.

\item Complexity level \textbf{L4:} four objects with four fine-resolution parts
  each; five rooms with six cells in each room.

\item Complexity level \textbf{L5:} eight objects with two fine-resolution parts
  each; five rooms with nine cells in each room.

\item Complexity level \textbf{L6:} eight objects with two fine-resolution parts
  each and four objects with one fine-resolution part each; five rooms with twelve cells in each room.

\item Complexity level \textbf{L7:} eight objects with two fine-resolution parts
  each and four objects with one fine-resolution part each; five rooms with sixteen cells in each room.

\item Complexity level \textbf{L8:} sixteen objects with two fine-resolution parts
  each and eight objects with one fine-resolution part each; five rooms with sixteen cells in each room.
\end{enumerate}


 
\end{document}
